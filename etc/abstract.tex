\chapter*{Sommario}
\label{cha:sommario}
\addcontentsline{toc}{chapter}{Sommario}

Questa tesi descrive il progetto di migrazione e parallelizzazione delle
procedure di installazione e aggiornamento del software di monitoraggio NetEye
presso l'azienda Würth Phoenix.\\ Il progetto è stato caratterizzato dall'adozione
della metodologia Agile, che ha svolto un ruolo centrale nel garantire una gestione
efficiente e senza intoppi delle varie fasi di lavoro, dall'analisi iniziale
alla realizzazione finale.\\ L'obiettivo principale del progetto è stato
sostituire un sistema obsoleto e poco manutenibile con una soluzione moderna e
idempotente basata su Ansible, migliorando l'efficienza e la manutenibilità delle
operazioni.\\ Un aspetto fondamentale del progetto è stata l'implementazione di una
logica di parallelizzazione per eseguire i vari step di configurazione in parallelo,
riducendo significativamente i tempi di esecuzione.\\ La metodologia Agile ha
permesso di gestire efficacemente lo sviluppo, il testing, la revisione del
codice e il rilascio, favorendo una stretta collaborazione e comunicazione tra i
membri del team.\\ Grazie a un approccio iterativo e incrementale, è stato
possibile affrontare e risolvere tempestivamente eventuali problemi, assicurando
il successo del progetto senza causare ritardi o inconvenienti.\\ I risultati ottenuti
hanno dimostrato un notevole miglioramento in termini di velocità, affidabilità
e facilità di manutenzione del sistema.\\ Il progetto ha permesso di ridurre i
tempi di installazione da minuti a secondi, garantendo al contempo la consistenza
e l'idempotenza delle operazioni.
\chapter{Introduzione}
\label{cha:introduzione}

\section{Introduzione generale al contesto del tirocinio}
\label{sec:tirocinio}

Il tirocinio è stato svolto presso Würth Phoenix, un'azienda con sede a Bolzano
specializzata nello sviluppo di soluzioni software per il monitoraggio e la
gestione delle reti aziendali.\\ Il principale prodotto di Würth Phoenix, NetEye,
è un software di monitoraggio delle reti utilizzato da aziende enterprise per
garantire l'efficienza e la sicurezza delle loro infrastrutture IT.\\ Lo scopo
del tirocinio da me svolto ha riguardato la migrazione del sistema di
installazione e aggiornamento di NetEye, da una soluzione obsoleta e poco mantenibile
a una più moderna e idempotente utilizzando Ansible.\\ Implementando inoltre una
logica di parallelizzazione per ottimizzare le procedure di installazione e aggiornamento
sono notevolmente migliorati i tempi di esecuzione e l'efficienza del sistema.\\
La metodologia Agile è stata una componente fondamentale del progetto,
facilitando la gestione delle fasi di sviluppo, la comunicazione e il coordinamento
tra i membri del team.\\ Ciò ha permesso di affrontare le sfide del progetto in modo
iterativo e incrementale, garantendo una maggiore flessibilità e adattabilità
alle esigenze in continua evoluzione.

\section{Obiettivi della tesi}
\label{sec:obiettivi}

La tesi si propone di illustrare in dettaglio il processo di migrazione e di parallelizzazione
delle procedure di installazione e aggiornamento di NetEye, sottolineando l'importanza
della metodologia Agile ed evidenziando le sfide affrontate e le soluzioni
adottate.\\ Gli obiettivi principali sono:
\begin{itemize}
  \item Descrivere il contesto e le motivazioni della migrazione: Esaminare il
    sistema di installazione e aggiornamento preesistente, evidenziandone le limitazioni
    e le problematiche che hanno reso necessaria la migrazione.

  \item Documentare il processo di migrazione ad Ansible: Spiegare come Ansible
    è stato utilizzato per creare un sistema di installazione e aggiornamento
    idempotente, illustrando i vantaggi ottenuti in termini di manutenibilità e affidabilità.

  \item Implementazione della parallelizzazione: Dettagliare il processo di
    sviluppo di un modulo Python per la parallelizzazione delle procedure di
    installazione e aggiornamento, descrivendo la logica di configurazione dei servizi
    e i benefici ottenuti in termini di efficienza.

  \item Analizzare i risultati ottenuti: Confrontare il sistema precedente con
    quello nuovo, valutando i miglioramenti in termini di efficienza,
    manutenibilità e idempotenza.

  \item Riflessioni sul lavoro di team e sulla metodologia Agile: Descrivere le
    dinamiche di lavoro in team, la comunicazione e il coordinamento tra i membri,
    e l'adozione della metodologia Agile durante il progetto.

  \item Suggerimenti per futuri miglioramenti: Offrire spunti e consigli per il
    miglioramento continuo del sistema e per future esperienze di tirocinio.
\end{itemize}

\section{Struttura della tesi}
\label{sec:struttura}

La tesi è strutturata nei seguenti capitoli:
\begin{enumerate}
  \item Introduzione: Presenta il contesto del tirocinio, gli obiettivi della tesi
    e la struttura del documento.

  \item Descrizione dell’Azienda e del Progetto: Descrive Würth Phoenix, il prodotto
    NetEye e i principali servizi offerti.

  \item Stato Iniziale del Sistema: Fornisce una descrizione dettagliata del
    sistema di installazione e aggiornamento originale, le tecnologie utilizzate,
    le procedure di installazione e aggiornamento di NetEye, e le necessità di migrazione
    a un sistema più moderno.

  \item Migrazione ad Ansible: Spiega il processo di migrazione, includendo la creazione
    del comando di installazione, la deprecazione del vecchio comando, l'aggiornamento
    delle pipelines di CI, delle immagini Docker, delle ISO e della User Guide,
    e la migrazione dei vecchi script.

  \item Parallelizzazione delle Procedure: Illustra il concetto e l'importanza
    della parallelizzazione, le fasi dell'implementazione, la configurazione dei
    servizi e la logica di dipendenze, e presenta esempi e risultati ottenuti.

  \item Metodologia Agile: Discute l'utilizzo della metodologia Agile, i ruoli e
    le fasi di sviluppo, e i metodi di comunicazione e coordinamento tra i
    membri del team.

  \item Collaborazione e Lavoro di Team: Esamina le dinamiche di lavoro in team,
    la comunicazione e il coordinamento, e l'utilizzo di strumenti Agile per il
    miglioramento continuo.

  \item Risultati e Conclusioni: Valuta i risultati del progetto di migrazione e
    parallelizzazione, confronta il sistema vecchio con quello nuovo, e offre
    riflessioni sull'esperienza di tirocinio e suggerimenti per futuri
    miglioramenti.
\end{enumerate}
Questa struttura permette di seguire un percorso logico e chiaro, che parte dal contesto
generale fino ad arrivare ai dettagli tecnici e alle conclusioni, fornendo una
visione completa del progetto svolto e dei risultati ottenuti.
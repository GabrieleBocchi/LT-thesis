\chapter{Metodologia Agile\cite{beck2001agile}}
\label{cha:agile}

\section{Introduzione alla metodologia Agile e ai suoi principi}
\label{sec:introduzione_agile}

La metodologia AGILE è caratterizzata da un insieme di principi e pratiche che mettono
in primo piano la flessibilità, la collaborazione e l'adattabilità ai
cambiamenti nella gestione dei progetti.\\ È stata sviluppata come risposta alle
limitazioni dei metodi tradizionali e si dimostra particolarmente idonea per
progetti di sviluppo software caratterizzati da requisiti in continua evoluzione,
dove la capacità di rispondere prontamente riveste un ruolo fondamentale.\\ Il Manifesto
AGILE\footnote{AGILE manifesto: \url{https://agilemanifesto.org}}, pubblicato nel
2001, descrive i valori fondamentali di questa metodologia: dando priorità alle interazioni
umane piuttosto che ai processi e agli strumenti, al software funzionante invece
di una documentazione esaustiva, alla collaborazione con il cliente anziché alla
negoziazione dei contratti e all'adattamento ai cambiamenti invece del seguire un
piano.

\section{Applicazione della metodologia Agile nel contesto del progetto}
\label{sec:applicazione_agile}

Durante il tirocinio volto al miglioramento delle procedure di NetEye, la metodologia
AGILE è stata applicata in modo centrale per garantire un processo lavorativo dinamico
e collaborativo.\\ La motivazione è stata principalmente la necessità di adattarsi
rapidamente ai cambiamenti e alle nuove sfide emerse durante lo sviluppo.\\ Inoltre
essendo un approccio già in uso dal team ha permesso l'integrazione del processo
di migrazione all'interno dei processi di sviluppo.

\subsection{Suddivisione del Lavoro in Sprint}
\label{sub:sprint}

Il lavoro è stato organizzato in \textit{Sprint}, ossia cicli di sviluppo brevi
e intensi di due settimane.\\ Ogni sprint aveva obiettivi specifici, definiti in
base alle priorità del progetto e alle esigenze del team.\\ Prima dell'inizio di
una sprint, si trova la fase di pianificazione, ovvero qualche ora dedicata a riunioni
volte a decidere gli obiettivi e le attività da completare. Il team decideva poi
quali task erano prioritarie e quali potevano essere rimandate alle sprint
successive.\\ Le due settimane successive alla pianificazione erano così
organizzate:
\begin{itemize}
  \item Daily Stand-Up Meetings: Brevi riunioni giornaliere durante le quali i
    membri del team si trovano per discutere sullo stato di avanzamento della
    sprint.\\ Ogni partecipante risponde a tre domande principali: Cosa ha fatto
    il giorno precedente, cosa farà oggi e se sono stati incontrati ostacoli che
    hanno rallentato il progresso.\\ Normalmente di 15 minuti, queste riunioni favoriscono
    una comunicazione scorrevole e permettono di individuare rapidamente eventuali
    problematiche.

  \item Sviluppo: È la fase in cui il team lavora sull'implementazione delle
    funzionalità definite nelle task assegnate durante la pianificazione della
    sprint.\\ Questa fase include la scrittura del codice singolarmente e in
    pair programming, i test unitari, l'integrazione del software e la
    risoluzione di bug.\\ Lo sviluppo avviene in cicli iterativi, con frequenti verifiche
    del progresso e adattamenti basati sui feedback ricevuti.

  \item Sprint Review: È una riunione che si tiene alla fine di ogni sprint, durante
    la quale il team presenta il lavoro completato agli stakeholders e agli altri
    sviluppatori.\\ L'obiettivo è dimostrare le funzionalità implementate e raccogliere
    feedback.\\ Questa riunione aiuta a garantire che il prodotto sviluppato soddisfi
    le aspettative del cliente e permette di fare aggiustamenti per gli sprint successivi
    in base alle osservazioni ricevute.

  \item Sprint Retrospective: È una riunione di riflessione che si svolge alla
    fine di ogni sprint, dopo la Sprint Review.\\ Durante questa riunione, il
    team discute cosa è andato bene, cosa potrebbe essere migliorato e cosa fare
    diversamente nella prossima sprint. L'obiettivo è identificare e
    implementare miglioramenti continui nel processo di lavoro, rafforzando le pratiche
    positive e correggendo quelle meno efficaci.
\end{itemize}

\subsection{Collaborazione e Comunicazione}
\label{sub:comunicazione}

La collaborazione tra i membri del team è stata facilitata dall'uso di strumenti
di comunicazione e gestione dei progetti, come Jira e Teams.\\ Questi strumenti
hanno permesso di mantenere una documentazione dettagliata del progresso del
progetto e di risolvere rapidamente eventuali problemi.
\begin{itemize}
  \item Jira: Utilizzato per tracciare le attività, assegnare compiti e monitorare
    lo stato di avanzamento di ogni sprint. Jira ha fornito una visione chiara e
    trasparente del progresso, facilitando la gestione delle priorità e l'assegnamento
    delle risorse.

  \item Teams: Ha facilitato la comunicazione istantanea tra i membri del team, permettendo
    di risolvere rapidamente domande e problemi senza dover attendere le riunioni
    formali.
\end{itemize}

\section{Benefici e sfide incontrate durante il progetto}
\label{sec:benefici_sfide_agile}

L'adozione della metodologia AGILE durante il progetto ha rappresentato un rafforzamento
significativo nel modo di lavorare del team. La flessibilità, la collaborazione e
l'attenzione al miglioramento continuo hanno permesso di affrontare le sfide del
progetto in modo efficace, garantendo il successo della migrazione.\\ Sebbene ci
siano state sfide da superare, i benefici ottenuti hanno dimostrato il valore di
questa metodologia per la gestione di progetti complessi e dinamici.

\subsection{Benefici}
\label{sub:benefici}

L'adozione della metodologia AGILE ha portato numerosi benefici al progetto di
migrazione e parallelizzazione di NetEye:
\begin{itemize}
  \item Maggiore flessibilità: La capacità di rispondere rapidamente ai
    cambiamenti e di adattare le priorità in base alle esigenze emergenti ha
    permesso di affrontare efficacemente le sfide del progetto.

  \item Migliore collaborazione: La comunicazione costante e la collaborazione
    tra i membri del team hanno ridotto i tempi di risoluzione dei problemi e
    migliorato la qualità del lavoro.

  \item Feedback continuo: Ha permesso di identificare e risolvere rapidamente
    eventuali problemi, migliorando costantemente il processo di sviluppo.

  \item Incrementi graduali: La suddivisione del lavoro in sprint ha permesso di
    realizzare incrementi graduali del software, riducendo i rischi e garantendo
    un progresso costante.
\end{itemize}

\subsection{Sfide}
\label{sub:sfide}

Tuttavia, l'implementazione della metodologia AGILE non è stata priva di sfide:
\begin{itemize}
  \item Coordinamento del team: Assicurare che tutti i membri del team fossero
    allineati sugli obiettivi e le priorità richiedeva un coordinamento costante
    e una comunicazione efficace.

  \item Gestione delle priorità: In un ambiente dinamico, la gestione delle priorità
    e delle risorse richiedeva una pianificazione attenta e la capacità di prendere
    decisioni rapide.

  \item Adattamento alla metodologia: L'adozione della metodologia AGILE ha
    richiesto un cambiamento di mentalità e di pratiche lavorative per tutti i
    membri del team, che non è stato sempre semplice.
\end{itemize}
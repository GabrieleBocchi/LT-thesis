\chapter{Stato Iniziale del Sistema}
\label{cha:stato_iniziale}

\section{Descrizione del vecchio sistema}
\label{sec:vecchio_sistema} NetEye è composto da una innumerevole quantità di
moduli e tecnologie per poter garantire un alto livello di affidabilità e
sicurezza; nel corso degli anni però, grazie principalmente all'aumentare delle
funzionalità, alcune delle tecnologie non risultavano più adeguate e si è mostrata
la necessità di migrare verso qualcosa di più mantenibile e scalabile.\\ È il caso
soprattutto delle procedure di installazione e aggiornamento, diventate un
enorme sequenza di script, i quali rendono molto complessa la gestione di queste
operazioni.\\ La necessità di cambiare è cresciuta sempre di più e questo è stato
il tema del tirocinio da me svolto.\\ Procediamo con la descrizione delle procedure
di installazione e aggiornamento allo stato precedente la migrazione e ai motivi
che hanno portato a questa decisione.

\subsection{Installazione e aggiornamento di NetEye}
\label{sub:install_update} La maggior parte dei software esistenti hanno in comune
principalmente due cose: l'installazione e l'aggiornamento.\\ Verranno ora
descritti i tre comandi che NetEye utilizza per gestire queste operazioni, verrà
inoltre fatta la distinzione fra l'esecuzione di un'installazione in una macchina
singola e quella invece in un cluster:
\begin{enumerate}
  \item \texttt{neteye\_secure\_install}: Questo comando si occupa di avviare una
    procedura di esecuzione seriale di script presenti in una determinata cartella.\\
    Per quanto riguarda l'installazione di NetEye:
    \begin{itemize}
      \item Single node: È sufficiente eseguire \texttt{neteye\_secure\_install}

      \item Cluster: È necessario mettere tutti i nodi in standby\footnote{Un
        nodo in standby è un nodo non operativo, ovvero nel quale non vengono eseguite
        risorse e servizi del cluster} ad eccezione del primo, poi eseguire
        serialmente \texttt{neteye\_secure\_install} su ogni nodo partendo dal primo,
        e poi rimuovere i nodi dallo stato di standby.
    \end{itemize}

  \item \texttt{neteye update}: Questo comando aggiorna la versione corrente di NetEye
    installando gli ultimi pacchetti rilasciati.\\

  \item \texttt{neteye upgrade}: Questo comando aggiorna NetEye alla versione successiva.
\end{enumerate}
Nelle procedure di update e upgrade non ci sono differenze riguardo a ciò che l'utente
deve fare in base al tipo di installazione, ma dietro le quinte viene eseguito nuovamente
il comando \texttt{neteye\_secure\_install}, portando quindi a un'incoerenza poiché
quest'ultimo dovrebbe logicamente essere eseguito solo la prima volta durante l'installazione
di NetEye.

\section{Limiti e necessità di migrazione a un'architettura più moderna}
\label{sec:necessita_di_migrazione} La complessità operativa e la difficoltà di manutenzione
rappresentavano uno dei principali problemi del sistema appena descritto.\\ Per gestire
efficacemente il software NetEye nella sua versione precedente, occorreva
possedere una notevole competenza tecnica. Spesso le operazioni di routine, come
l'installazione e l'aggiornamento, erano inefficienti e complicate, richiedendo
molte fasi manuali che aumentavano la possibilità di commettere errori.\\ A
causa dell'assenza di un'architettura idempotente, era complicato assicurarsi che
le operazioni ripetute producessero sempre gli stessi risultati, rendendo così
ancora più complessa la gestione e la risoluzione dei problemi.\\ Una grande
limitazione in fase di sviluppo invece, era l'aggiunta di nuovi script per configurare
servizi o eseguire operazioni, dovuta al fatto che era molto complesso capire se
implementando qualcosa di nuovo potesse andare in errore qualcosa di già
esistente.\\ Il vecchio sistema era caratterizzato da comandi lenti, che creavano
un'inefficienza operativa significativa. Le configurazioni di nuovi dispositivi
o l'aggiornamento di essi spesso richiedevano tempi d'attesa prolungati.\\ Oltre
a ridurre la produttività, questi ritardi potevano mettere a rischio anche l'identificazione
tempestiva e la soluzione dei problemi di rete, aumentando il pericolo di
downtime e interruzioni del servizio.\\ \\ Si è previsto che la migrazione a un
sistema moderno e idempotente comporterà numerosi vantaggi. Tra questi:
\begin{itemize}
  \item Maggiore Efficienza Operativa: Per ottenere una maggiore velocità nell'esecuzione
    dei comandi e semplificare le operazioni ricorrenti.

  \item Scalabilità Migliorata: Il sistema può essere ampliato e gestito senza
    che ciò influisca sulle prestazioni.

  \item Esperienza Utente Ottimizzata: I comandi sono stati migliorati e
    semplificati per ridurre la difficoltà di apprendimento e aumentare l'efficienza.

  \item Adattabilità e Innovazione: Una maggiore facilità nell'incorporare nuove
    tecnologie e adattarsi alle mutevoli esigenze del mercato è una
    caratteristica apprezzabile.
\end{itemize}
Per riassumere, la richiesta di passare a un sistema più moderno è giustificata da
una serie di elementi che includono l'efficienza delle operazioni, la
possibilità di espandere il sistema, l'esperienza dell'utente e la capacità di introdurre
nuove idee.\\ L'implementazione di una soluzione è cruciale per superare le limitazioni
del sistema precedente e assicurare un futuro migliore allo sviluppo e all'utilizzo
di NetEye.
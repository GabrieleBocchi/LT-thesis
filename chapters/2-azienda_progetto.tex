\chapter{Descrizione dell'Azienda e del Progetto}
\label{cha:azienda_progetto}

\section{Panoramica dell'azienda}
\label{sec:panoramica} Würth Phoenix è un'azienda fondata nel 2000 con sede a
Bolzano specializzata in soluzioni per la gestione e il monitoraggio delle reti
aziendali.\\ Parte dell'internazionale gruppo Würth, l'obiettivo di Würth Phoenix
è fornire strumenti avanzati e affidabili che assistano le aziende nel controllo
e l'ottimizzazione delle loro infrastrutture tecnologiche, prestando attenzione
a sicurezza, visibilità e manutenibilità.\\ L'azienda offre una serie di prodotti
e servizi che vanno a coprire le diverse aree informatiche delle aziende clienti,
migliorandone le prestazioni, la sicurezza e la conformità.\\ Grazie alla forte attenzione,
all'innovazione e alla qualità del lavoro svolto, Würth Phoenix ha ottenuto nel tempo
una solida reputazione nel settore IT, servendo una vasta gamma di clienti, dalle
piccole e medie imprese, fino alle grandi corporazioni multinazionali.\\ Inoltre,
le collaborazioni con partner tecnologici internazionali hanno fortemente contribuito
al miglioramento dei prodotti e servizi che l'azienda offre, aumentandone l'efficienza
d'uso e riuscendo a soddisfare al meglio le esigenze dei clienti.

\section{Servizi e prodotti}
\label{sec:servizi_prodotti} Würth Phoenix offre servizi che spaziano dalle
business application grazie a Microsoft Dynamics 365, servizi di threat intelligence
e cybersecurity, fino a soluzioni di monitoraggio e systems management grazie al
principale prodotto dell'azienda: NetEye\\

\subsection{Descrizione del software NetEye}
\label{sub:neteye} NetEye è il principale prodotto di Würth Phoenix, sviluppato
per poter fornire un monitoraggio completo e versatile delle reti aziendali.\\ È
una piattaforma che combina una moltitudine di diverse funzionalità in un unico
sistema, offrendo agli amministratori della rete, una visione chiara e
dettagliata dello stato dell'infrastruttura informatica dell'azienda.\\ NetEye
può monitorare una vasta gamma di componenti, inclusi server, dispositivi di
rete di qualunque genere, applicazioni e servizi.\\ È disponibile sia in versione
on-premise, quindi direttamente nelle aziende dei clienti nella loro
infrastruttura, sia in cloud, con il servizio NetEye Cloud, equivalente alla prima
opzione, ma che non richiede al cliente un'infrastruttura in azienda, se non delle
piccole macchine chiamate "satelliti" che inviano tutti i dati a NetEye Cloud per
essere processati e fornire tutti i report necessari.\\ Inoltre l'installazione di
NetEye è disponibile in due versioni, single node, il che significa che il software
è installato in una sola macchina, o cluster, installato quindi in più macchine per
poter garantire una potenza di calcolo molto più elevata e un monitoraggio migliore.\\
Può essere considerata a tutti gli effetti una distro Linux, poiché viene
installata una versione custom di RHEL, ovvero RedHat Enterprise Linux,
direttamente nelle macchine finali.\\ Inoltre, è un prodotto estremamente modulare,
ogni cliente può installare i feature modules che preferisce e utilizzare tutte
le funzioni che mettono a disposizione.\\ La parte core di NetEye, quindi senza feature
modules installati, fornisce principalmente monitoraggio, è composta da \href{https://github.com/Icinga/icinga2}{ICINGA2}
e \href{https://github.com/Icinga/icingaweb2}{Icinga Web 2}, soluzioni open
source sviluppate da Icinga Project, di cui Würth Phoenix è partner, e da
\href{https://github.com/WuerthPhoenix/tornado}{Tornado}, un Event Processor che
riceve dati da una fonte arbitraria e dopo averli analizzati, esegue azioni ed
eventi di conseguenza.\\ Verranno ora analizzati i feature modules disponibili
su NetEye:
\begin{itemize}
  \item Alyvix: e' un application performance monitoring, permette di monitorare
    performance di applicazioni web e desktop SU SOLO WINDOWS

  \item Asset Management: GLPI, ticketing ecc

  \item Command Orchestrator: monitoraggio si integra con Icinga, puoi fare azioni
    su macchine remote, guardare user guide

  \item SIEM: elastic stack, con le 5 cose + El Proxy, guardare user guide

  \item ntopng: monitoraggio di rete, open source, cercare su internet

  \item SLM: roba nostra, e' un modulo Icinga Web 2 (Service Level Management), puoi
    definire contratti utilizzati per accertare una certa SLA

  \item vSphereDB: chiedere a GIAN o MATTIA
\end{itemize}
NetEye è multitenant, il che significa che l'infrastruttura monitorata può
essere suddivisa in parti più piccole per poter dare, ai giusti dipendenti o supervisori,
visibilità solo su ciò che è di loro competenza, si può facilmente immaginare con
i settori di un'azienda, gestiti da dipendenti diversi.\\ cluster HA, HA si basa
su pcs(redhat) e drbd(LINBIT), spiegarli. Importanza della HA, risorse che si
spostano in base al carico, PCS sposta mentre drbd e' il filesystem replicato e
distribuito, si accerta che i servizi abbiamo le configurazioni ecc
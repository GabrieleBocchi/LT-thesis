\chapter{Descrizione dell'Azienda e del Progetto}
\label{cha:azienda_progetto}

\section{Panoramica dell'azienda}
\label{sec:panoramica} Würth Phoenix è un'azienda fondata nel 2000 con sede a
Bolzano specializzata in soluzioni per la gestione e il monitoraggio delle reti
aziendali.\\ Parte dell'internazionale gruppo Würth, l'obiettivo di Würth Phoenix
è fornire strumenti avanzati e affidabili che assistano le aziende nel controllo
e l'ottimizzazione delle loro infrastrutture tecnologiche, prestando attenzione
a sicurezza, visibilità e manutenibilità.\\ L'azienda offre una serie di prodotti
e servizi che vanno a coprire le diverse aree informatiche delle aziende clienti,
migliorandone le prestazioni, la sicurezza e la conformità.\\ Grazie alla forte attenzione,
all'innovazione e alla qualità del lavoro svolto, Würth Phoenix ha ottenuto nel tempo
una solida reputazione nel settore IT, servendo una vasta gamma di clienti, dalle
piccole e medie imprese, fino alle grandi corporazioni multinazionali.\\ Inoltre,
le collaborazioni con partner tecnologici internazionali hanno fortemente contribuito
al miglioramento dei prodotti e servizi che l'azienda offre, aumentandone l'efficienza
d'uso e riuscendo a soddisfare al meglio le esigenze dei clienti.

\section{Servizi e prodotti}
\label{sec:servizi_prodotti} Würth Phoenix offre servizi che spaziano dalle
business application grazie a Microsoft Dynamics 365, servizi di threat intelligence
e cybersecurity, fino a soluzioni di monitoraggio e systems management grazie al
principale prodotto dell'azienda: NetEye.\\

\subsection{Descrizione del software NetEye}
\label{sub:neteye} NetEye è il principale prodotto di Würth Phoenix, sviluppato per
poter fornire un monitoraggio completo e versatile delle reti aziendali.\\ È una
piattaforma che combina una moltitudine di diverse funzionalità in un unico sistema,
offrendo agli amministratori della rete una visione chiara e dettagliata dello
stato dell'infrastruttura informatica dell'azienda.\\ NetEye può monitorare una
vasta gamma di componenti, inclusi server, dispositivi di rete di qualunque
genere, applicazioni e servizi.\\ È disponibile sia in versione on-premise,
quindi direttamente nelle aziende dei clienti, sia in cloud, con il servizio
NetEye Cloud che è equivalente alla prima opzione, ma che non richiede al cliente
un'infrastruttura in azienda ad eccezione di piccole macchine chiamate "satelliti"
che inviano tutti i dati a NetEye Cloud per poi essere processati e fornire tutti
i report necessari. Grazie a questo sistema il cliente non si deve occupare
della gestione e della manutenzione di tutta l'infrastruttura.\\ In aggiunta, l'installazione
di NetEye è disponibile in due versioni: single node, il che significa che il software
è installato in una sola macchina; e cluster, installato quindi in più macchine per
poter garantire una potenza di calcolo molto più elevata e un monitoraggio migliore.\\
Può essere considerata a tutti gli effetti una distribuzione Linux poiché viene installata
una versione custom di RHEL, ovvero RedHat Enterprise Linux, direttamente nelle macchine
finali.\\ Inoltre è un prodotto estremamente modulare, ogni cliente può
installare i moduli aggiuntivi che preferisce e utilizzare tutte le funzioni che
mettono a disposizione.\\ La parte core di NetEye, dunque senza moduli aggiuntivi
installati, fornisce principalmente monitoraggio; è composta da \href{https://github.com/Icinga/icinga2}{Icinga
2} e \href{https://github.com/Icinga/icingaweb2}{Icinga Web 2}, soluzioni open
source sviluppate da Icinga Project, di cui Würth Phoenix è partner, e da
\href{https://github.com/WuerthPhoenix/tornado}{Tornado}, un Event Processor che
riceve dati da una fonte arbitraria e dopo averli analizzati, esegue azioni ed
eventi di conseguenza.\\ Verranno ora analizzati i moduli aggiuntivi disponibili
su NetEye: \\----------------------------TODO----------------------------
\begin{itemize}
  \item Alyvix: TODO è un application performance monitoring, permette di monitorare
    performance di applicazioni web e desktop SU SOLO WINDOWS

  \item Asset Management: TODO GLPI, ticketing ecc

  \item Command Orchestrator: TODO monitoraggio si integra con Icinga, puoi fare
    azioni su macchine remote, guardare user guide

  \item SIEM: TODO elastic stack, con le 5 cose + El Proxy, guardare user guide

  \item ntopng(Network Top Next Generation): È un software di monitoraggio di reti
    open-source creato da ntop. Offre un'interfaccia web facile da utilizzare
    per monitorare in tempo reale il flusso del traffico di rete. ntopng consente
    di visualizzare informazioni dettagliate, inclusi l'utilizzo della larghezza
    di banda, le applicazioni in uso, i flussi di dati e gli indirizzi IP
    coinvolti sfruttando tecnologie avanzate. È in grado inoltre di supportare diversi
    protocolli di rete e può essere facilmente integrato con altri strumenti di monitoraggio
    per fornire una visualizzazione completa delle prestazioni e della sicurezza
    della rete, offrendo nel contempo informazioni dettagliate.\\ ntopng viene
    utilizzato per ottimizzare la gestione della rete, identificare anomalie e potenziali
    minacce, nonché migliorare la pianificazione delle risorse di rete grazie
    alla sua avanzata capacità di analisi.

  \item SLM(Service Level Management): Modulo Icinga Web 2 sviluppato internamente,
    è possibile definire dei contratti tra l'azienda Würth Phoenix e un
    determinato cliente che vengono utilizzati per accertare una certa SLA(Service
    Level Agreement), ovvero un indicatore riguardo il tempo che un certo
    servizio o risorsa erano operativi.

  \item vSphereDB: TODO
\end{itemize}
----------------------------TODO----------------------------\\ NetEye è multitenant,
il che significa che l'infrastruttura monitorata può essere suddivisa in parti
più piccole per poter dare, ai giusti dipendenti o supervisori, visibilità solo
su ciò che è di loro competenza, si può facilmente immaginare con i settori di un'azienda,
gestiti da dipendenti diversi.\\ Infine, i cluster di NetEye sono High Availability,
ciò si riferisce a una strategia di progettazione del sistema e alla sua
implementazione che assicura un certo livello di prestazioni operative e tempo
di funzionamento, solitamente superiore al 99,99\%.\\ Su NetEye, ciò viene garantito
da due software essenziali:
\begin{itemize}
  \item PCS(Pacemaker/Corosync Cluster Stack): È uno strumento di gestione dei cluster
    sviluppato da Red Hat e utilizzato per configurare e controllare cluster High
    Availability.\\ Si compone principalmente di due componenti:
    \begin{itemize}
      \item Pacemaker: Si occupa della gestione delle risorse di un cluster. Il suo
        compito principale consiste nel supervisionare lo stato di esse e nell'eseguire
        automaticamente il ripristino in caso di guasti.\\ Pacemaker garantisce la
        disponibilità continua delle risorse, servizi o applicazioni, tramite la
        redistribuzione in caso di guasto del nodo. Gestisce le relazioni tra le
        risorse e prende decisioni sul loro avvio o arresto in base a criteri
        predefiniti.

      \item Corosync: Ha il compito di gestire la comunicazione tra i nodi. Offre
        un sistema di messaggistica stabile e si assicura che tutti i nodi siano
        sempre informati sullo stato del cluster. Si occupa della gestione della
        membership, del quorum e del monitoraggio dei nodi.
    \end{itemize}
    In sintesi, PCS è utilizzato per configurare, gestire e mantenere cluster
    High Availability.\\ Si assicura che le risorse siano distribuite
    correttamente e che il sistema possa riprendersi rapidamente da eventuali
    guasti.

  \item DRBD(Distributed Replicated Block Device): È una soluzione di replicazione
    dei dati di un file system per Linux sviluppata da LINBIT che consente la replica
    di dischi tra diversi nodi.\\ È utilizzato per garantire la disponibilità e l'integrità
    dei dati in ambienti clusterizzati.
\end{itemize}
Utilizzando assieme PCS e DRBD si riesce a garantire la creazione di sistemi
robusti che possono resistere a guasti hardware senza interruzioni significative
dei servizi. Un funzionamento di questo tipo è essenziale per i sistemi che utilizzano
NetEye e che hanno bisogno di un monitoraggio continuo.
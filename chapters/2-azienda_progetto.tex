\chapter{Descrizione dell'Azienda e del Progetto}
\label{cha:azienda_progetto}

\section{Panoramica dell'azienda}
\label{sec:panoramica} Würth Phoenix è un'azienda fondata nel 2000 con sede a
Bolzano specializzata in soluzioni per la gestione e il monitoraggio delle reti
aziendali.\\ Parte dell'internazionale gruppo Würth, l'obiettivo di Würth Phoenix
è fornire strumenti avanzati e affidabili che assistano le aziende nel controllo
e l'ottimizzazione delle loro infrastrutture tecnologiche, prestando attenzione
a sicurezza, visibilità e manutenibilità.\\ L'azienda offre una serie di prodotti
e servizi che vanno a coprire le diverse aree informatiche delle aziende clienti,
migliorandone le prestazioni, la sicurezza e la conformità.\\ Grazie alla forte attenzione,
all'innovazione e alla qualità del lavoro svolto, Würth Phoenix ha ottenuto nel tempo
una solida reputazione nel settore IT, servendo una vasta gamma di clienti, dalle
piccole e medie imprese, fino alle grandi corporazioni multinazionali.\\ Inoltre,
le collaborazioni con partner tecnologici internazionali hanno fortemente contribuito
al miglioramento dei prodotti e servizi che l'azienda offre, aumentandone l'efficienza
d'uso e riuscendo a soddisfare al meglio le esigenze dei clienti.

\section{Servizi e prodotti}
\label{sec:servizi_prodotti} Würth Phoenix offre servizi che spaziano dalle
business application grazie a Microsoft Dynamics 365, servizi di threat intelligence
e cybersecurity, fino a soluzioni di monitoraggio e systems management grazie al
principale prodotto dell'azienda: NetEye.\\

\subsection{Descrizione del software NetEye}
\label{sub:neteye} NetEye è il principale prodotto di Würth Phoenix, sviluppato per
poter fornire un monitoraggio completo e versatile delle reti aziendali.\\ È una
piattaforma che combina una moltitudine di diverse funzionalità in un unico sistema,
offrendo agli amministratori della rete una visione chiara e dettagliata dello
stato dell'infrastruttura informatica dell'azienda.\\ NetEye può monitorare una
vasta gamma di componenti, inclusi server, dispositivi di rete di qualunque
genere, applicazioni e servizi.\\ È disponibile sia in versione on-premise,
quindi direttamente nelle aziende dei clienti, sia in cloud, con il servizio
NetEye Cloud che è equivalente alla prima opzione, ma che non richiede al cliente
un'infrastruttura in azienda ad eccezione di piccole macchine chiamate "satelliti"
che inviano tutti i dati a NetEye Cloud per poi essere processati e fornire tutti
i report necessari. Grazie a questo sistema il cliente non si deve occupare
della gestione e della manutenzione di tutta l'infrastruttura.\\ In aggiunta, l'installazione
di NetEye è disponibile in due versioni: single node, il che significa che il software
è installato in una sola macchina; e cluster, installato quindi in più macchine per
poter garantire una potenza di calcolo molto più elevata e un monitoraggio High Availability,
ciò si riferisce a una strategia di progettazione del sistema e alla sua
implementazione per cui il sistema continua a funzionare anche nei casi in cui
alcuni nodi dovessero avere problemi e che assicura un certo livello di
prestazioni operative e tempo di funzionamento, solitamente superiore al 99,99\%.\\
Può essere considerata a tutti gli effetti una distribuzione Linux poiché viene
installata una versione custom di RHEL 8.10\footnote{RedHat Enterprise Linux
website: \url{https://www.redhat.com/en/technologies/linux-platforms/enterprise-linux}},
ovvero RedHat Enterprise Linux, direttamente nelle macchine finali.\\ Inoltre è un
prodotto estremamente modulare, ogni cliente può installare i moduli aggiuntivi
che preferisce e utilizzare tutte le funzioni che mettono a disposizione.\\ La
parte core di NetEye, dunque senza moduli aggiuntivi installati, fornisce
principalmente monitoraggio; è composta da Icinga 2\footnote{Icinga 2 source
code: \url{https://github.com/Icinga/icinga2}} e Icinga Web 2\footnote{Icinga
Web 2 source code: \url{https://github.com/Icinga/icingaweb2}}, soluzioni open
source sviluppate da Icinga Project, di cui Würth Phoenix è partner, e da
Tornado\footnote{Tornado source code: \url{https://github.com/WuerthPhoenix/tornado}},
un Event Processor che riceve dati da una fonte arbitraria e dopo averli
analizzati, esegue azioni ed eventi di conseguenza.\\ Verranno ora analizzati i moduli
aggiuntivi disponibili su NetEye:
\begin{itemize}
  \item Alyvix\footnote{Alyvix website: \url{https://alyvix.com}}: È uno strumento
    avanzato di visual monitoring che si distingue per la sua capacità di eseguire
    test automatizzati delle interfacce delle applicazioni.\\ Utilizzando tecniche
    di riconoscimento delle immagini, Alyvix simula le interazioni dell'utente
    con l'interfaccia grafica, consentendo un monitoraggio accurato delle performance
    delle applicazioni.\\ Ciò permette di rilevare in tempo reale eventuali anomalie
    o rallentamenti, assicurando che le applicazioni funzionino sempre in modo
    ottimale.\\ È in grado di generare script di test che imitano le azioni
    degli utenti, come cliccare su pulsanti, inserire dati nei campi di input e
    navigare tra diverse schermate. Questi test possono essere programmati per essere
    eseguiti a intervalli regolari, fornendo un monitoraggio costante e
    dettagliato.\\ Alyvix, inoltre, offre report molto approfonditi, permettendo
    agli amministratori di sistema di visualizzare i risultati dei test e di identificare
    rapidamente i problemi.\\ L'integrazione di Alyvix con NetEye consente di sfruttare
    le potenti funzionalità di monitoraggio di quest'ultimo, potendo gestire con
    dati dettagliati e facilmente interpretabili tutta l'infrastruttura di rete
    di un'azienda.

  \item Asset Management: Questo modulo si basa principalmente sull'uso di GLPI\footnote{GLPI
    website: \url{https://glpi-project.org}} (Gestionnaire Libre de Parc
    Informatique), un potente strumento di gestione degli asset IT.\\ GLPI offre
    una piattaforma centralizzata per la gestione dettagliata e il monitoraggio
    dei componenti hardware e software.\\ Grazie a questo strumento, è possibile
    tenere traccia dello stato, della posizione e delle proprietà degli asset, facilitando
    una gestione accurata e aggiornata del parco informatico aziendale Oltre a GLPI,
    il modulo di gestione degli asset di NetEye utilizza OCS Inventory (Open Computer
    and Software Inventory Next Generation) per la scoperta automatizzata degli
    asset e l'inventario.\\ OCS Inventory scansiona la rete per rilevare e
    raccogliere dati sulle configurazioni hardware e software, riducendo la necessità
    di inserimenti manuali e minimizzando gli errori.

  \item Command Orchestrator: Questo è uno strumento fondamentale per l'automazione
    e il monitoraggio delle operazioni IT.\\ Integrato con Icinga, il Command Orchestrator
    consente di eseguire azioni su macchine remote in modo coordinato e automatizzato.
    Questa funzionalità è essenziale per mantenere l'efficienza operativa e
    garantire che tutte le componenti del sistema IT siano sotto controllo.\\
    Grazie all'integrazione con Icinga, il Command Orchestrator può monitorare in
    tempo reale lo stato delle varie risorse IT, come server, applicazioni e servizi.\\
    Quando viene rilevato un problema o un'anomalia, il sistema può eseguire
    automaticamente dei comandi predefiniti per risolvere l'incidente o per
    attivare procedure di mitigazione. Questo approccio riduce i tempi di inattività
    e migliora la reattività del team IT.\\ Un altro aspetto cruciale del Command
    Orchestrator è la capacità di eseguire azioni su macchine remote.\\ Questo significa
    che gli amministratori di sistema possono automatizzare operazioni complesse,
    come aggiornamenti software, riavvii di servizio, backup di dati, e altre attività
    di manutenzione, su una vasta gamma di dispositivi distribuiti in diverse
    sedi geografiche. L'automazione di queste operazioni non solo migliora l'efficienza
    ma riduce anche il rischio di errori umani.

  \item SIEM: Questo modulo utilizza Elastic Stack\footnote{Elastic website: \url{https://www.elastic.co}}
    ed El Proxy per offrire una soluzione completa per la gestione e l'analisi degli
    eventi di sicurezza.\\ Verranno ora elencati e illustrati tutti i software di
    questo stack:
    \begin{itemize}
      \item Elasticsearch: Motore di ricerca distribuito e un sistema di analisi
        in tempo reale basato su Apache Lucene\footnote{Apache Lucene source
        code: \url{https://github.com/apache/lucene}}. È progettato per gestire
        grandi quantità di dati e supportare ricerche complesse attraverso un'architettura
        distribuita.

      \item Logstash: Strumento per la raccolta, processo e distribuzione dei log.\\
        Riceve i dati da diverse fonti e li invia a Elasticsearch o ad altri sistemi
        di archiviazione e analisi.\\ Supporta numerosi input e output, inclusi
        file di log, database, metriche di rete, e offre la possibilità di trasformare
        i dati prima dell'invio.

      \item Kibana: Interfaccia utente che consente di visualizzare e analizzare
        i dati archiviati in Elasticsearch.\\ È progettato per la
        visualizzazione dei dati in tempo reale attraverso grafici, mappe,
        tabelle e dashboard personalizzabili.\\ Inoltre facilita l'esplorazione
        e l'interpretazione dei dati attraverso una serie di strumenti di
        visualizzazione interattivi.

      \item Beats: Agenti leggeri utilizzati per inviare dati operativi a
        Elasticsearch o a Logstash.\\ Esistono diversi tipi di Beats
        specializzati in diverse tipologie di dati, come metriche di sistema,
        log di rete o dati di audit.\\ Ciò semplifica la raccolta dei dati da
        sorgenti distribuite, riducendo il carico di lavoro sui sistemi di raccolta
        e analisi.

      \item Elastic Agent: È un'evoluzione di Beats che unifica l'esperienza di
        raccolta dei dati in singoli agenti, offrendo la possibilità di raccogliere
        dati, applicare politiche di sicurezza e integrità, e inviare i dati a Elasticsearch
        o altri sistemi di destinazione. Nonostante gli agenti possano essere
        remoti, tutto questo può essere gestito centralmente dall'interfaccia di
        Kibana.\\ Non è necessario specificare il tipo di Beat da utilizzare per
        una certa task, è invece sufficiente indicare le operazioni necessarie
        affinché venga automaticamente selezionato il Beat più idoneo.\\ Elastic
        Agent è progettato per semplificare la gestione degli agenti e
        migliorare le prestazioni e la sicurezza delle operazioni di
        monitoraggio e analisi dei dati.

      \item El Proxy: Soluzione interamente sviluppata da Würth Phoenix che si
        occupa di garantire l'integrità e la sequenzialità dei log prima della
        loro indicizzazione in Elasticsearch.\\ Questo sistema firma i log in
        ingresso, creando una blockchain che assicura i log non vengano alterati
        e che la loro sequenza sia mantenuta in modo corretto.\\ Questo livello di
        sicurezza e integrità è essenziale per aziende che devono rispettare requisiti
        normativi rigorosi sulla conservazione e la gestione dei dati.\\ EL Proxy
        offre quindi una soluzione robusta per le aziende che necessitano di un metodo
        sicuro e affidabile per la gestione dei loro log, assicurando che ogni
        evento registrato sia autentico e non alterato.
    \end{itemize}

  \item ntopng(Network Top Next Generation): È un software di monitoraggio di reti
    open-source creato da ntop\footnote{ntop website: \url{https://www.ntop.org}}.\\
    Offre un'interfaccia web facile da utilizzare per monitorare in tempo reale
    il flusso del traffico di rete.\\ ntopng consente di visualizzare
    informazioni dettagliate, inclusi l'utilizzo della larghezza di banda, le
    applicazioni in uso, i flussi di dati e gli indirizzi IP coinvolti sfruttando
    tecnologie avanzate.\\ È in grado inoltre di supportare diversi protocolli di
    rete e può essere facilmente integrato con altri strumenti di monitoraggio per
    fornire una visualizzazione completa delle prestazioni e della sicurezza della
    rete, offrendo nel contempo informazioni dettagliate.\\ ntopng viene
    utilizzato per ottimizzare la gestione della rete, identificare anomalie e potenziali
    minacce, nonché migliorare la pianificazione delle risorse di rete grazie
    alla sua avanzata capacità di analisi.

  \item SLM(Service Level Management): Questo modulo, sviluppato internamente
    come modulo per Icinga Web 2, è uno strumento cruciale per la gestione e la verifica
    delle SLA (Service Level Agreements). Ciò rappresenta il parametro chiave
    negli accordi tra un'azienda e i suoi clienti, specificando gli standard di servizio
    che devono essere mantenuti.\\ Il modulo SLM consente di definire contratti dettagliati
    che stabiliscono i livelli di servizio attesi per varie risorse o servizi.
    Questi contratti includono metriche chiave come la disponibilità e le
    prestazioni, che vengono monitorate costantemente per garantire che gli standard
    concordati siano rispettati.

  \item vSphereDB\footnote{vSphereDB source code: \url{https://github.com/Icinga/icingaweb2-module-vspheredb}}:
    È un modulo open-source di Icinga Web 2 progettato per il monitoraggio completo
    delle macchine virtuali su piattaforme VMware vSphere.\\ Permette agli amministratori
    di sistema di avere una visione dettagliata e centralizzata di tutte le VM gestite
    tramite vSphere.\\ Il modulo funziona tramite un demone che accede all'infrastruttura
    vSphere e raccoglie una vasta gamma di dati sulle macchine virtuali. Questo
    include informazioni sullo stato delle VM, le prestazioni, l'utilizzo delle
    risorse e molto altro.\\ I dati raccolti vengono poi visualizzati in un'interfaccia
    utente intuitiva all'interno di NetEye.
\end{itemize}
NetEye è multitenant, il che significa che l'infrastruttura monitorata può essere
suddivisa in parti più piccole per poter dare, ai giusti dipendenti o
supervisori, visibilità solo su ciò che è di loro competenza, si può facilmente
immaginare con i settori di un'azienda, gestiti da manager diversi.\\ Come specificato
precedentemente i cluster di NetEye sono High Availability, e ciò viene
garantito da due software essenziali:
\begin{itemize}
  \item PCS(Pacemaker/Corosync Cluster Stack): È uno strumento di gestione dei cluster
    sviluppato da Red Hat e utilizzato per configurare e controllare cluster High
    Availability.\\ Si compone principalmente di due componenti:
    \begin{itemize}
      \item Pacemaker: Si occupa della gestione delle risorse di un cluster. Il suo
        compito principale consiste nel supervisionare lo stato di esse e nell'eseguire
        automaticamente il ripristino in caso di guasti.\\ Pacemaker garantisce la
        disponibilità continua delle risorse, servizi o applicazioni, tramite la
        redistribuzione in caso di guasto del nodo. Gestisce le relazioni tra le
        risorse e prende decisioni sul loro avvio o arresto in base a criteri
        predefiniti.

      \item Corosync: Ha il compito di gestire la comunicazione tra i nodi. Offre
        un sistema di messaggistica stabile e si assicura che tutti i nodi siano
        sempre informati sullo stato del cluster. Si occupa della gestione della
        membership, del quorum e del monitoraggio dei nodi.
    \end{itemize}
    In sintesi, PCS è utilizzato per configurare, gestire e mantenere cluster
    High Availability.\\ Si assicura che le risorse siano distribuite
    correttamente e che il sistema possa riprendersi rapidamente da eventuali
    guasti.

  \item DRBD(Distributed Replicated Block Device): È una soluzione di replicazione
    dei dati di un file system per Linux sviluppata da LINBIT che consente la replica
    di dischi tra diversi nodi.\\ È utilizzato per garantire la disponibilità e l'integrità
    dei dati in ambienti clusterizzati.
\end{itemize}
Utilizzando assieme PCS e DRBD si riesce a garantire la creazione di sistemi robusti
che possono resistere a guasti hardware senza interruzioni significative dei servizi.
Un funzionamento di questo tipo è essenziale per i sistemi che utilizzano NetEye
e che hanno bisogno di un monitoraggio continuo.
\chapter{Collaborazione e Lavoro di Team}
\label{cha:team}

\section{Dinamiche di lavoro in team durante il tirocinio}
\label{sec:introduzione_team}

\subsection{Ruoli e responsabilità}
\label{sub:ruoli}

La definizione chiara dei ruoli e delle responsabilità presso Würth Phoenix è
stata un elemento cruciale durante il tirocinio.\\ Ciascun membro del gruppo
aveva incarichi specifici assegnati in base alle proprie abilità e conoscenze, assicurando
così un flusso di lavoro efficace e ben organizzato.\\ I ruoli più importanti dei
vari membri del team erano i seguenti:
\begin{itemize}
  \item Sviluppo: La prima fase, quella della scrittura del codice.\\ Durante lo
    sviluppo vengono implementate nuove funzionalità, corretti bug o
    semplicemente migliorate le prestazioni attuali del prodotto.\\ La
    documentazione del codice e delle soluzioni adottate è parte integrante di
    questa fase.

  \item Testing: Questa è una fase critica che mira a garantire che il software
    funzioni correttamente e sia privo di bug.\\ Ci sono molte tipologie di test,
    da quelle delle singole funzionalità che garantiscono che queste ultime funzionino
    correttamente, a quelle più generiche riguardanti tutto il prodotto che
    controllano che includendo tali modifiche tutto il complesso sistema
    funzioni ancora come previsto.\\ I risultati dei test vengono documentati e
    analizzati per migliorare continuamente la qualità del prodotto.

  \item Controllo del codice: Chiamata anche \textit{code review}, è la fase di
    revisione del codice scritto dagli sviluppatori da parte di altri membri del
    team.\\ Questo passaggio è fondamentale per mantenere alti standard di
    qualità, rilevare errori o inefficienze e condividere conoscenze all'interno
    del team.\\ Inizialmente la code review è fatta in automatico da strumenti che
    garantiscono che il codice sia scritto secondo le linee guida aziendali,
    dopodiché ci sono i controlli manuali da parte degli altri sviluppatori, e infine
    è richiesta una security review che mira ad accertare l'assenza di vulnerabilità
    nelle modifiche introdotte.

  \item Approvazione degli stakeholder: Questo è uno dei momenti più importanti,
    in cui le nuove funzionalità e modifiche vengono presentate agli stakeholder
    per il loro feedback e la loro approvazione.\\ Possono essere incluse demo,
    presentazioni e discussioni dettagliate per assicurarsi che il prodotto
    soddisfi le aspettative e i requisiti dei clienti e degli utenti finali.

  \item Scrittura delle Release Notes: Questa è l'ultima fase del processo di
    sviluppo in cui vengono documentate tutte le nuove funzionalità, i miglioramenti
    e i bug risolti in una nuova versione del software.\\ Le release notes sono destinate
    agli utenti finali e agli stakeholder per informarli delle modifiche apportate
    e di come queste influenzeranno l'uso del prodotto.\\ Di estrema importanza è
    rendere le release notes chiare e dettagliate, fornendo anche istruzioni su
    come utilizzare le nuove funzionalità e su come risolvere eventuali problemi
    noti.
\end{itemize}

\subsection{Metodologie di lavoro}
\label{sub:metodologie_lavoro}

Durante il tirocinio, diverse metodologie di lavoro sono state adottate per garantire
un processo di sviluppo efficiente e di alta qualità. Queste metodologie hanno
permesso al team di collaborare efficacemente, affrontare le sfide in modo sistematico
e migliorare continuamente i processi interni.\\ Una delle metodologie chiave è stata
il Pair Programming, che coinvolge due sviluppatori che lavoravano insieme sullo
stesso pezzo di codice. Questo approccio ha facilitato la condivisione delle conoscenze,
migliorato la qualità del codice e ridotto gli errori.\\ Un'altra componente
fondamentale per la riuscita di questo progetto è stata la metodologia Agile, che
ha permesso un'ottima organizzazione del lavoro, garantendo che tutti i membri
del team fossero allineati sugli obiettivi e che le risorse fossero utilizzate
in modo ottimale.

\section{Comunicazione e coordinamento tra i membri del team}
\label{sec:coordinamento}

La comunicazione e il coordinamento tra i membri del team sono stati cruciali per
il successo del progetto.\\ Diversi metodi e strumenti sono stati utilizzati per
garantire che tutti i membri del team fossero sempre allineati e che le decisioni
potessero essere prese in modo collaborativo ed efficace.\\ Oltre alla comunicazione
tramite Teams, discussa nel punto \ref{sub:comunicazione}, erano frequenti anche
le riunioni di persona.\\ Questi incontri, che avvenivano in ufficio, erano particolarmente
utili per discussioni approfondite e per prendere decisioni importanti.\\ Le riunioni
faccia a faccia permettevano una migliore interazione e comprensione reciproca,
essenziale per affrontare argomenti complessi e per mantenere un buon rapporto
di lavoro all'interno del team.\\ Importantissimi per le decisioni tecniche e di
progetto erano gli LLD (Low-Level Design), ovvero delle riunioni registrate e più
formali in cui si discute l'implementazione tecnica di una nuova funzionalità che
si vuole introdurre. Gli LLD permettevano di discutere e validare le scelte
progettuali in modo strutturato, riducendo il rischio di errori e malintesi.
Infine, un altro aspetto fondamentale era l'aiuto reciproco.\\ I membri del team
si sostenevano a vicenda, condividendo conoscenze e competenze per superare le
sfide tecniche e migliorare continuamente.\\ Questo spirito di collaborazione ha
permesso di affrontare i problemi in modo più efficace e di trovare soluzioni
innovative. Il supporto reciproco è stato un fattore chiave per il miglioramento
delle competenze individuali e per il rafforzamento del team nel suo insieme.

\section{Utilizzo di strumenti Agile per il miglioramento continuo}
\label{sec:strumenti_agile}

\subsection{Implementazione di Kanban e Scrum}
\label{sub:kanban_scrum}

La combinazione delle metodologie Kanban e Scrum ha permesso di gestire il flusso
di lavoro in modo efficiente e flessibile, adattandosi alle esigenze del
progetto e garantendo una consegna continua e di qualità.:
\begin{itemize}
  \item Kanban: Utilizzato per visualizzare e gestire il flusso di lavoro
    continuo.\\ Le bacheche Kanban hanno permesso di monitorare lo stato delle
    attività, identificare i colli di bottiglia e migliorare il processo di sviluppo.\\
    Questa metodologia ha aiutato a mantenere un flusso di lavoro costante e a ridurre
    i tempi di attesa tra le diverse fasi del progetto.\\ Grazie alla visualizzazione
    delle attività in corso, è stato possibile ottimizzare le risorse e
    garantire che ogni membro del team fosse sempre impegnato in attività
    produttive, minimizzando così i tempi morti.

  \item Scrum: Implementato tramite cicli di sprint di due settimane. Ogni sprint
    includeva una fase di pianificazione, esecuzione, review e retrospective.\\ Questo
    approccio iterativo ha permesso di adattarsi rapidamente ai cambiamenti e di
    migliorare continuamente il prodotto.\\ Durante gli sprint, il team ha
    potuto concentrarsi su obiettivi chiari e raggiungibili, mentre le review e le
    retrospective hanno fornito opportunità per riflettere sui risultati ottenuti,
    apprendere dagli errori e implementare miglioramenti continui.
\end{itemize}
L'integrazione di Kanban e Scrum ha quindi garantito un equilibrio tra la
gestione del flusso di lavoro continuo e l'adattabilità necessaria per
rispondere alle mutevoli esigenze del progetto, permettendo al team di mantenere
un alto livello di produttività e qualità nel tempo.

\subsection{Feedback e adattamento}
\label{sub:feedback}

La raccolta e l'integrazione continua dei feedback sono stati cruciali per il
miglioramento del progetto.\\ I feedback venivano raccolti in diverse occasioni,
tra cui la fine delle sprint e durante le usuali fasi di sviluppo.\\ L'adattamento
alle nuove situazioni e alle lezioni apprese era una parte integrante del ciclo di
miglioramento continuo. Ad esempio, se durante una retrospective emergeva che un
particolare metodo di comunicazione non era efficace, il team sperimentava nuove
soluzioni per migliorare la trasparenza e la collaborazione.\\ Questo poteva
significare l'introduzione di nuovi strumenti, la modifica delle modalità di riunione,
o l'adozione di nuove pratiche di gestione del progetto.
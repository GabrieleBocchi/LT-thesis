\chapter{Migrazione ad Ansible}
\label{cha:migrazione}

\section{Introduzione ad Ansible e ai suoi vantaggi}
\label{sec:ansible}

Ansible è un potente software open-source sviluppato da Red Hat che si occupa di
automazione IT.\\ Distinguendosi per la sua semplicità ed efficienza, è
progettato per migliorare e facilitare la gestione della configurazione e del
provisioning delle applicazioni, oltre che l'orchestrazione e l'automazione delle
operazioni informatiche più note.\\ Grazie ad Ansible, gli amministratori di
sistema e i DevOps riescono ad automatizzare facilmente compiti complessi.\\ Una
caratteristica fondamentale di Ansible è la sua natura agentless, il che comporta
l'assenza della necessità di installare software specifico sui nodi target.
Questo garantisce un significativo abbattimento della complessità e dei
requisiti di manutenzione, mentre si verificano contemporaneamente miglioramenti
in termini di sicurezza e affidabilità delle operazioni.\\ Il linguaggio utilizzato
per scrivere i playbook, ovvero i file che descrivono ciò che Ansible dovrà
eseguire nei nodi target, è YAML, un linguaggio facilmente leggibile sia dagli
umani che dalle macchine.\\ Attraverso l'utilizzo dei playbook, è possibile fornire
una descrizione completa delle attività da eseguire, come la configurazione dei
server, l'installazione di pacchetti software, la gestione degli utenti e dei
permessi e molte altre operazioni.\\ L'architettura di Ansible si basa su moduli
che possono essere utilizzati per eseguire una vastissima gamma di operazioni, ed
è inoltre possibile scriverne di custom, per poter aggiungere funzionalità più
specifiche.\\ Un'altra caratteristica vantaggiosa di Ansible è l'idempotenza, che
garantisce la coerenza e la ripetibilità delle operazioni, indipendentemente
dallo stato iniziale del sistema. In ambienti IT dinamici e complessi come questo,
la coerenza e l'affidabilità delle configurazioni sono essenziali.\\ In breve, Ansible
fornisce un'opzione efficace e versatile per automatizzare le operazioni IT,
ottimizzando l'efficienza dei processi, diminuendo gli errori umani e consentendoci
di risparmiare tempo prezioso.\\ L'adozione di questa soluzione può portare notevoli
benefici in termini di scalabilità e produttività delle infrastrutture.

\section{Processo di migrazione}
\label{sec:processo_migrazione}

Analizzeremo ora le fasi e le motivazioni che hanno portato alla migrazione da
quello che era un sistema basato sull'esecuzione seriale di script a un sistema
di playbook per garantire una configurazione idempotente e mantenibile del
sistema.

\subsection{Verso un sistema idempotente}
\label{sub:prerequisiti_migrazione}

Durante la fase decisionale riguardo il miglioramento delle procedure di NetEye,
una delle caratteristiche più importanti che si è voluto ottenere è stata l'idempotenza.\\
Questo perché, in un ambiente complesso come quello in cui opera NetEye, è
fondamentale garantire la coerenza e la ripetibilità delle operazioni, indipendentemente
dallo stato iniziale del sistema.\\ Per ottenere tutto ciò si sono dovute analizzare
minuziosamente le caratteristiche e le fasi delle varie procedure.

\subsection{Fasi della migrazione}
\label{sub:fasi_migrazione}

Il processo di migrazione è stato tutt'altro che banale, spesso quando un problema
sembrava risolto ne veniva alla luce un altro, è stato necessario dunque
ricorrere a un approccio strutturato e metodico.\\ Inizialmente il lavoro non è
stato altro che riunioni e discussioni per decidere come procedere, ma alla fine
si è riusciti a redigere un piano operativo per riuscire efficientemente e,
soprattutto, senza creare problemi a procedere per questo primo passo verso il miglioramento
delle procedure di NetEye:
\begin{enumerate}
  \item Creare il comando \texttt{neteye install}:

  \item Deprecare il comando \texttt{neteye\_secure\_install}:

  \item Aggiornare le pipelines della continuous integration:

  \item Aggiornare le immagini Docker:

  \item Aggiornare le ISO:

  \item Aggiornare la User Guide:

  \item Migrare i vecchi script:
\end{enumerate}

\section{Implementazione di un nuovo sistema}
\label{sec:nuovo_sistema}

\subsection{Metodo di sviluppo}
\label{sub:sviluppo}

Modificare una parte così centrale di un software grandissimo come NetEye è stata
una sfida veramente complessa ma estremamente stimolante, essenziali però sono
stati gli strumenti messi a disposizione e che hanno affiancato lo sviluppo
rendendo fattibile l'aggiunta e la modifica di codice, verranno ora brevemente illustrati:
\begin{itemize}
  \item Jira:

  \item Jenkins:

  \item OpenShift:

  \item Cluster Provisioner:

  \item vSphere:
\end{itemize}

\subsection{Problematiche affrontate durante la migrazione}
\label{sub:problemi}
\chapter{Migrazione ad Ansible}
\label{cha:migrazione}

\section{Introduzione ad Ansible e ai suoi vantaggi}
\label{sec:ansible}

Ansible\footnote{Ansible website: \url{https://www.ansible.com}} è un potente
software open-source sviluppato da Red Hat che si occupa di automazione IT.\\
Distinguendosi per la sua semplicità ed efficienza, è progettato per migliorare e
facilitare la gestione della configurazione e del provisioning delle applicazioni,
oltre che l'orchestrazione e l'automazione delle operazioni informatiche più
note.\\ Grazie ad Ansible, gli amministratori di sistema e i DevOps riescono ad automatizzare
facilmente compiti complessi.\\ Una caratteristica fondamentale di Ansible è la sua
natura agentless, il che comporta l'assenza della necessità di installare
software specifico sui nodi target. Questo garantisce un significativo abbattimento
della complessità e dei requisiti di manutenzione, mentre si verificano
contemporaneamente miglioramenti in termini di sicurezza e affidabilità delle
operazioni.\\ Il linguaggio utilizzato per scrivere i playbook, ovvero i file che
descrivono ciò che Ansible dovrà eseguire nei nodi target, è YAML, un linguaggio
facilmente leggibile sia dagli umani che dalle macchine.\\ Attraverso l'utilizzo
dei playbook, è possibile fornire una descrizione completa delle attività da
eseguire, come la configurazione dei server, l'installazione di pacchetti
software, la gestione degli utenti e dei permessi e molte altre operazioni.\\ L'architettura
di Ansible si basa su moduli che possono essere utilizzati per eseguire una vastissima
gamma di operazioni, ed è inoltre possibile scriverne di custom, per poter aggiungere
funzionalità più specifiche.\\ Un'altra caratteristica vantaggiosa di Ansible è l'idempotenza,
che garantisce la coerenza e la ripetibilità delle operazioni, indipendentemente
dallo stato iniziale del sistema. In ambienti IT dinamici e complessi come
questo, la coerenza e l'affidabilità delle configurazioni sono essenziali.\\ In breve,
Ansible fornisce un'opzione efficace e versatile per automatizzare le operazioni
IT, ottimizzando l'efficienza dei processi, diminuendo gli errori umani e
consentendoci di risparmiare tempo prezioso.\\ L'adozione di questa soluzione
può portare notevoli benefici in termini di scalabilità e produttività delle
infrastrutture.

\section{Processo di migrazione}
\label{sec:processo_migrazione}

Analizzeremo ora le fasi e le motivazioni che hanno portato alla migrazione da quello
che era un sistema basato sull'esecuzione seriale di script a un sistema di playbook
per garantire una configurazione idempotente e mantenibile del sistema.

\subsection{Verso un sistema idempotente}
\label{sub:prerequisiti_migrazione}

Durante la fase decisionale riguardo il miglioramento delle procedure di NetEye,
una delle caratteristiche più importanti che si è voluto ottenere è stata l'idempotenza.\\
Questo perché, in un ambiente complesso come quello in cui opera NetEye, è fondamentale
garantire la coerenza e la ripetibilità delle operazioni, indipendentemente
dallo stato iniziale del sistema.\\ Per ottenere tutto ciò si sono dovute
analizzare minuziosamente le caratteristiche e le fasi delle varie procedure.

\subsection{Fasi della migrazione}
\label{sub:fasi_migrazione}

Il processo di migrazione è stato tutt'altro che banale, spesso quando un
problema sembrava risolto ne veniva alla luce un altro, è stato necessario dunque
ricorrere a un approccio strutturato e metodico.\\ Inizialmente è stato un lungo
lavoro di pianificazione per decidere come procedere, ma alla fine si è riusciti
a redigere un piano operativo per riuscire efficientemente e, soprattutto, senza
fermare lo sviluppo degli altri membri del team, a procedere per questo primo passo
verso il miglioramento delle procedure di NetEye:
\begin{enumerate}
  \item Creare il comando \texttt{neteye install}:
    \begin{itemize}
      \item Descrizione: Come prima operazione, è stato creato il comando \texttt{neteye
        install} da affiancare temporaneamente al comando \texttt{neteye\_secure\_install},
        con l'obiettivo di sostituire quest'ultimo al termine della migrazione.

      \item Funzionamento: Questo nuovo comando esegue serialmente i playbook
        Ansible presenti in una specifica cartella del sistema. Inizialmente,
        \texttt{neteye install} è stato inserito all'inizio della vecchia procedura,
        permettendo la migrazione graduale degli script mantenendo intatta la
        procedura complessiva.

      \item Obiettivo: Garantire una transizione senza interruzioni, permettendo
        di verificare il corretto funzionamento del nuovo comando prima di deprecare
        quello vecchio.
    \end{itemize}

  \item Deprecare il comando \texttt{neteye\_secure\_install}:
    \begin{itemize}
      \item Descrizione: Una volta verificato il corretto funzionamento di
        \texttt{neteye install}, il comando \texttt{neteye\_secure\_install} è stato
        deprecato.

      \item Transizione: Inizialmente, \texttt{neteye install} era posizionato
        all'inizio del \texttt{neteye\_secure\_install}. Successivamente, \texttt{neteye\_secure\_install}
        è stato spostato alla fine di \texttt{neteye install}, rendendo
        superfluo il vecchio comando per l'utente.

      \item Obiettivo: Eliminare il vecchio comando, semplificando la procedura di
        installazione e riducendo la complessità.
    \end{itemize}

  \item Aggiornare le pipelines della Continuous Integration:
    \begin{itemize}
      \item Descrizione: A seguito della modifica del comando di installazione, è
        stato necessario aggiornare tutte le pipeline di testing per utilizzare il
        nuovo comando.

      \item Obiettivo: Garantire che il nuovo processo di installazione sia
        completamente integrato e testato attraverso le pipeline di CI, mantenendo
        l'affidabilità e la qualità del software.
    \end{itemize}

  \item Aggiornare le immagini Docker:
    \begin{itemize}
      \item Descrizione: Le immagini Docker, utilizzate per testare versioni containerizzate
        di NetEye, sono state aggiornate nei Dockerfile per riflettere le nuove
        procedure di installazione.

      \item Obiettivo: Assicurare che le immagini Docker siano compatibili con
        il nuovo comando di installazione, permettendo un testing e un deployment
        coerenti in ambienti containerizzati.
    \end{itemize}

  \item Aggiornare le ISO:
    \begin{itemize}
      \item Descrizione: Le immagini ISO utilizzate per installare NetEye su
        macchine fisiche per i clienti sono state aggiornate per includere il
        nuovo processo di installazione automatizzato.

      \item Obiettivo: Fornire una soluzione di installazione aggiornata e
        ottimizzata per i server fisici, migliorando l'esperienza di deployment per
        i clienti.
    \end{itemize}

  \item Aggiornare la User Guide:
    \begin{itemize}
      \item Descrizione: La guida utente è stata aggiornata per documentare
        accuratamente le nuove procedure di installazione, fornendo ai clienti istruzioni
        chiare e dettagliate.

      \item Obiettivo: Assicurare che gli utenti abbiano accesso a una
        documentazione aggiornata e precisa, facilitando l'adozione delle nuove pratiche
        di installazione.
    \end{itemize}

  \item Migrare i vecchi script:
    \begin{itemize}
      \item Descrizione: Una volta completata la migrazione del comando di
        installazione, è iniziata la conversione dei vecchi script di installazione
        e configurazione in playbook Ansible.

      \item Processo: Questa fase richiede tempo, poiché il numero di script da migrare
        è elevato. Il processo prevede una revisione dettagliata e la
        riscrittura di ogni script per garantire la compatibilità e l'idempotenza.

      \item Obiettivo: Completare la transizione verso un sistema completamente
        automatizzato e manutenibile, migliorando la coerenza e l'affidabilità delle
        installazioni.
    \end{itemize}
\end{enumerate}

\section{Implementazione di un nuovo sistema}
\label{sec:nuovo_sistema}

\subsection{Metodo di sviluppo}
\label{sub:sviluppo}

Modificare una parte così centrale di un software grandissimo come NetEye è stata
una sfida veramente complessa ma estremamente stimolante, essenziali sono stati
gli strumenti messi a disposizione e che hanno affiancato lo sviluppo rendendo
fattibile l'aggiunta e la modifica di codice.\\ Verranno ora brevemente
illustrati:
\begin{itemize}
  \item Jira: Strumento di gestione dei progetti e di tracciamento dei bug
    sviluppato da Atlassian.\\ È utilizzato principalmente per il monitoraggio
    dei problemi, la gestione dei progetti Agile, e il controllo del flusso di
    lavoro.\\ Jira supporta le metodologie Scrum e Kanban, consentendo ai team di
    pianificare, tracciare e rilasciare software di alta qualità.\\ Gli utenti
    possono creare backlog, sprint, e storie utente, oltre a visualizzare il progresso
    tramite bacheche Kanban.

  \item Jenkins: Servizio open-source di automazione che facilita i processi di
    \textit{Continuous Integration} e \textit{Continuous Delivery} (CI/CD) nel
    processo di sviluppo software.\\ Jenkins automatizza la costruzione, il testing
    e la distribuzione del codice, permettendo agli sviluppatori di rilevare
    rapidamente i difetti e integrare nuove funzionalità.\\ Con la sua vasta
    gamma di plugin, Jenkins può essere integrato con numerosi strumenti di sviluppo
    e gestione del ciclo di vita del software.

  \item OpenShift: Piattaforma di containerizzazione e orchestrazione sviluppata
    da Red Hat, basata su Kubernetes\footnote{\label{kubernetes_note}Kubernetes source
    code: \url{https://github.com/kubernetes/kubernetes}}.\\ OpenShift fornisce
    un ambiente integrato per lo sviluppo, la distribuzione e la gestione delle applicazioni
    containerizzate.\\ Offre funzionalità avanzate come la gestione automatica dei
    cluster, il bilanciamento del carico, il monitoraggio e il ridimensionamento
    automatico, supportando una vasta gamma di linguaggi e framework di sviluppo.

  \item Cluster Provisioner: Strumento utilizzato per creare e gestire cluster
    di server in modo automatizzato.\\ Questo tipo di software è fondamentale
    per la configurazione di ambienti di sviluppo, testing e produzione,
    garantendo che i cluster siano configurati correttamente e pronti per l'uso.\\
    I Cluster Provisioner possono essere utilizzati in ambienti cloud, on-premises
    o ibridi, e spesso supportano l'integrazione con strumenti di orchestrazione
    come Kubernetes\footref{kubernetes_note}.

  \item vSphere: Piattaforma di virtualizzazione di VMware che consente di
    creare, eseguire e gestire macchine virtuali su larga scala.\\ vSphere include
    un hypervisor (VMware ESXi) per l'esecuzione delle VM e una suite di strumenti
    per la gestione delle risorse di elaborazione, rete e storage.\\ vSphere è
    ampiamente utilizzato in ambienti di data center per migliorare l'efficienza
    operativa, ridurre i costi e aumentare la flessibilità e la scalabilità delle
    infrastrutture IT.
\end{itemize}

\subsection{Problematiche affrontate durante la migrazione}
\label{sub:problemi}

Non sono mancati i problemi durante le fasi della migrazione.\\ Poiché NetEye è un
prodotto molto esteso e pieno di funzionalità, le difficoltà e il timore di
creare problemi inizialmente non hanno contribuito al miglioramento delle
procedure.\\ Con il passare dei giorni però la consapevolezza riguardo al
funzionamento di NetEye è aumentata ed è stato possibile procedere con la
migrazione.\\ Inoltre, durante le fasi transitorie si sono presentati numerosi problemi
dovuti al fatto che si stavano modificando procedure che gli altri sviluppatori usavano
da anni, ed è quindi stato necessario creare documenti e video al fine di
documentare correttamente le modifiche, per gli attuali e futuri membri del team
che lavoreranno con esse.
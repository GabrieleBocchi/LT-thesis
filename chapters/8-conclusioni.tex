\chapter{Risultati e Conclusioni}
\label{cha:conclusioni}

\section{Valutazione dei risultati del progetto}
\label{sec:valutazione}

Il progetto ha riguardato due aspetti principali: la migrazione dei vecchi
script ad Ansible e l'implementazione della parallelizzazione nelle procedure di
NetEye.\\ Entrambi gli obiettivi sono stati raggiunti con successo, apportando miglioramenti
significativi in termini di efficienza, manutenibilità e idempotenza:
\begin{itemize}
  \item Migrazione ad Ansible: La migrazione ha comportato la riscrittura delle
    procedure di installazione e aggiornamento utilizzando playbook Ansible, il che
    ha permesso di ottenere configurazioni idempotenti e facilmente replicabili.\\
    La modularità di Ansible ha reso le procedure più facili da mantenere e aggiornare,
    riducendo la complessità e aumentando la robustezza del sistema.

  \item Parallelizzazione: L'implementazione della parallelizzazione ha
    ulteriormente migliorato l'efficienza delle procedure.\\ Grazie al modulo
    sviluppato in Python, è stato possibile eseguire in parallelo i servizi indipendenti,
    riducendo significativamente i tempi di installazione da 2 minuti a soli 20
    secondi.
\end{itemize}

\section{Confronto tra il sistema vecchio e quello nuovo}
\label{sec:confronto}

Per meglio comprendere i vantaggi ottenuti, è utile confrontare il sistema
vecchio con quello nuovo in termini di efficienza, manutenibilità e idempotenza:
\begin{itemize}
  \item Efficienza: Il vecchio sistema richiedeva tempi di installazione molto
    lunghi, mentre il nuovo sistema, grazie alla parallelizzazione, ha drasticamente
    ridotto i tempi di esecuzione.\\ La stessa procedura di installazione che prima
    richiedeva 2 minuti ora viene completata in 20 secondi. Questo miglioramento
    ha un impatto significativo sulla produttività e sulla soddisfazione del
    cliente.

  \item Manutenibilità: La migrazione ad Ansible ha semplificato enormemente la
    manutenzione del sistema.\\ La struttura modulare dei playbook Ansible e la
    possibilità di gestire le configurazioni tramite file YAML rendono le
    procedure di installazione molto più gestibili e facili da aggiornare
    rispetto ai vecchi script.\\ Inoltre gli aggiornamenti e le modifiche
    possono essere effettuati in modo più rapido e con meno rischi di errori.

  \item Idempotenza: Il vecchio sistema non garantiva l'idempotenza, il che poteva
    portare a configurazioni incoerenti e difficili da risolvere.\\ Con Ansible,
    ogni playbook è idempotente per natura, garantendo che l'esecuzione ripetuta
    delle stesse operazioni porti sempre allo stesso stato finale,
    indipendentemente dallo stato iniziale.\\ Questo aumenta l'affidabilità e la
    prevedibilità delle installazioni.
\end{itemize}

\section{Conclusioni tratte dall’esperienza di tirocinio e suggerimenti per
futuri miglioramenti}
\label{sec:suggerimenti}

L'adozione della metodologia Agile, già in uso all'interno del team, ha
facilitato la gestione del progetto e ha permesso una rapida adattabilità ai
cambiamenti.\\ La migrazione ad Ansible e l'implementazione della
parallelizzazione sono state esperienze altamente formative, che hanno dimostrato
l'importanza dell'innovazione e dell'automazione nei processi aziendali.\\ Suggerimenti
tecnici:
\begin{itemize}
  \item Documentazione accurata: Tenere una documentazione accurata del proprio
    lavoro è fondamentale per garantire la continuità e per facilitare la
    comprensione del progetto a nuovi membri del team.

  \item Ottimizzazione continua della parallelizzazione: Sebbene i risultati
    ottenuti siano stati molto positivi, ci sono ancora margini di miglioramento
    nella gestione delle dipendenze e nella configurazione dei servizi.\\ Un'ulteriore
    ottimizzazione della logica di parallelizzazione potrebbe portare a riduzioni
    ancora maggiori nei tempi di esecuzione.

  \item Monitoraggio e logging avanzato: Implementare un sistema di monitoraggio
    e logging più avanzato per le procedure di installazione e aggiornamento
    potrebbe aiutare a identificare e risolvere più rapidamente eventuali
    problemi, migliorando ulteriormente l'affidabilità del sistema.
\end{itemize}
Suggerimenti sul fattore Agile e umano:
\begin{itemize}
  \item Collaborazione attiva: Partecipare attivamente alle riunioni di team e
    ai daily stand-up per rimanere sempre aggiornati sugli sviluppi del progetto
    e per contribuire con idee e feedback.

  \item Formazione continua: Approfittare delle opportunità di formazione e
    delle risorse aziendali per migliorare continuamente le proprie competenze
    tecniche e metodologiche.

  \item Proattività: Essere proattivi nel proporre soluzioni e miglioramenti, dimostrando
    iniziativa e spirito di innovazione.

  \item Feedback e miglioramento: Utilizzare il feedback ricevuto durante il
    tirocinio per migliorare continuamente il proprio lavoro e adattarsi alle
    esigenze del progetto e del team. In conclusione, il progetto ha dimostrato
    come l'adozione di tecnologie moderne e metodologie agili possa portare a
    significativi miglioramenti in termini di efficienza, manutenibilità e affidabilità.\\
    L'esperienza acquisita durante il tirocinio sarà preziosa per affrontare future
    sfide e per continuare a migliorare i processi aziendali.
\end{itemize}